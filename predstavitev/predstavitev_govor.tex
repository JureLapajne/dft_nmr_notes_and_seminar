\documentclass[openany, longbibliography,slovene,a4paper,12pt]{book}
\usepackage[a4paper,inner=3.5cm,outer=2.5cm,top=2.5cm,bottom=2.5cm]{geometry}

\usepackage{braket}
\usepackage{float}
\usepackage{afterpage}
\usepackage{graphicx}
\usepackage{amssymb}

\usepackage[tbtags]{amsmath}
\usepackage[T1]{fontenc}
\graphicspath{{../slike/}{../slike_vezikel_z_robom/}{/home/jure/sola/magisterij/uporabljene_slike/}
{../eps_pdf/}}
\DeclareGraphicsExtensions{.eps,.jpeg,.png,.gif,.pdf}
\usepackage[outdir=../uporabljene_slike/]{epstopdf}
\epstopdfsetup{
	suffix=,
}
\usepackage[multidot]{grffile}

\usepackage[slovene]{babel}      % slovenski delilni vzorci (!)
\usepackage[utf8]{inputenc}
\usepackage{makeidx}
\usepackage{enumerate}
\usepackage{caption}
\usepackage{subcaption}
\usepackage[tbtags]{mathtools}

\usepackage[section]{placeins}

\usepackage[hyphens,spaces,obeyspaces]{url}
\usepackage{breakurl}


\usepackage{ragged2e}
\edef\UrlBreaks{\do\-\UrlBreaks}

\usepackage{makeidx}
\pagestyle{headings}
\makeindex
\usepackage{fancyhdr}
\usepackage[titletoc,title]{appendix}


\usepackage[sort, numbers]{natbib}
\usepackage[pdfa]{hyperref}
\usepackage[x-1a]{pdfx}
\usepackage{pdfpages}


\begin{document}
\section{naslovna stran}
Pozdravljeni,
danes vam bom predstavil moj magisterij z naslovom liidni vesikli v valjasti ograditvi, ki sem ga pripavil pod mentorstvom prof dr. primoža ziherla.
\section{slide1}
Najprej si poglejmo kaj sploh so lipidni vesikli. Lipidni vesikli so strukture sestavljene iz fosfolipidnega dvosloja. Le-ta je sestavljen iz amfifilnih molekul. Ko molekule raztopimo v topilu, ki je običajno voda, se molekule začno združevati različne skupke, v našem primeru v dvosloj. To se zgodi zato, ker je interakcija med polarnimi repi in vodo energijsko zelo neugodna. Polarni repi se zato postavijo tako, da niso v stiku z vodo. Zaplate dvoslojev še vedno omogočajo stik repov z vodo (pokaži na sliki). Zato zaplate dvoslojev niso končno stanje. Tvorijo se sklenjene strukture, ki jim rečemo vesikli. Študij vesiklov je pomemben za razumevanje membrane. Razumevanje slednjih pa je pomembno za razumevanje zgradbe in delovanja celic in celičnih organelov. Membrana slednjih je pravtako sestavljena iz fosfolipidnega dvosloja, vendar vsebuje še mnoge druge moelekule, ki so pomembne za obliko in funkcijo membrane. Na slednjo poleg kemijske sestave vpliva tudi vsebina celice oz. organela in citoskelet.

\section{slide2}
V tem delu sem se podrobneje osredotočil na mitohodrij. Mitohondrij ima dve membrani. Notranja tvori številne uvihke, ki jih lahko vidimo na slikah. Na zgornji sliki uvihki izgledajo kot globoke, ozke invaginacije, vendar njihove 3 dimenzionalne strukture ne moremo videti. Spodnji sliki sta elektronska tomograma, na katerih vidimo 3 dimenzionalno strukturo uvihkov. Uvihki so lahko tubularni ali lamelarni. Glede na prevladujočo obliko ločimo dve stanji mitohondrija: ortodoksno in kondenzirano. Za prvo je značilno intenzivno celično dihanje in tubularni uvihki, za drugo pa počasno celično dihanje in lamelarni uvihki. Za njihov nastanek sicer obstaja več razlag, najverjetnejša pravi, da so uvihki posledica prostorske ograditve notranje membrane, ki ima večjo površino od zunanje.

\section{slide3}
Ograjeni vesikli, ki imajo površino večjo od ograditve so že bili proučevani. Na zgornji sliki lahko vidimo ekspreimentalno dobljene lipidne vesikle, na spodnji pa numerično izračunane oblike. Vidimo, da se nekatere oblike lepo ujemajo z eksperimentalnimi (pokaži, 1,2,3,4).

\section{slide4}
Preprost teoretičen model je leta 2014 v svojem magisteriju predstavil Bor Kavčič. Vesikle je sestavil iz preprostih geometrijski oblik kot so sfere, valjasti deli, ravne stene, za katere lahko na preprost način izračunamo njihovo površino, volumen, energijo,... Dobljeni analitični izrazi dajo boljši vpogled na bistvene odvisnosti. Model dobro velja v približki tesne ograditve, kar pomeni, da je volumen vesikla zelo blizu volumna ograditve. Namen tega dela je razširitev modela na vesikle v valjasti ograditvi.

\section{slide5}
Oglejmo si parametre s akterimi opišemo vesikle. Opišemo jih relativno glede na sfero z enako površino. Zato definiramo polmer $R_s$, ki je polmer sfere z enako površino kot jo ima vesikel. REducirana prostornina je definirana kot razmerje med prostornino vesikla in prostornino sfere z enako površino. Reducirana razlika površin monoslojev je definirana kot razmerje med razliko površin monoslojev vesikla in sfere. Razlika površin monoslojev je razlika površine med zgornjim slojem in spodnjim slojem dvosloja membrane. Večja ukrivljenost poveča razliko površin. Lahko je pozitivna ali negativna.

\section {slide6}
Podobno, relativno glede na sfero, definiramo tudi brezdimenzijsko energijo vesikla. Le-ta ima 3 prispevke. Lokalno upogibno enegijo nam podaja ta (in pokažeš) izraz, kjer je H lokalna povrprečna ukrivljenost membrane. Drugi prispevek je medmembranska interakcija, kjer $\gamma$ podaja moč in vrsto interakcije: odbojna oz. privlačna. Tretji prispevek je nelokalna upogibna energija, ki temelji na reducirani razliki površin monoslojev. $\Delta a_0$ je ravnovesna vrdnost reducirane razlike površine monoslojev.

\section{slide7}
Poglejmo si sedaj modelske oblike. Na levi sliki zgoraj vidimo ograditev, ki nima končnih ploskev. Torej je kot neskončno dolga valjasta cev. Obliko vesikla v cevi smo modelirali z zunanjo obliko, in uvihki. Zunanja oblika sestoji iz valjastega dela in pokrovov na konceh. Membranske uvihke so sestavljeni iz telesa uvihka in strukture, s katero se pripenjajo na zunanjo obliko. Tem strukturam pravimo robovi. Na tem mestu vpeljemo še polnilno razmerje, ki pove kako zelo se volumen vesikla razlikuje od volumna ograditve. V približku tesne ograditve mora biti polnilno razmerje blizu 1. Z drugimi besedami to pomeni da mora biti $r$ veliko manjši od polmera $r$.

\section{slide 8}
Kot že omenjeno poznamo več različnih vrst robov. Robove je vpeljal že Bor v svojem delu leta 2014. Tu (in pokažeš na skico) vidimo trojni rob, ki ga sicer nismo uporabili, smo pa uporabili zunanji rob, ki je le psoebna oblika trojnega robu. Zunanji rob dobimo iz trojnega obu tako da enega izmed robov $\alpha$, $\beta$ ali $\gamma$ postavimo na $\pi$. Naslednji rob je $U$, ki običajno povezuje dve tesno prilegajoči se plasti membrane. Zadnji uporabljeni rob je torusni rob, ki ga najdemo na mestih, kjer tubularni uvihki izraščajo iz zuannje oblike.

\section{slide 10}
Iz zgoraj predstavljeni oblik smo sestavili modelske oblike. Na povsem levi sliki vidimo vesikel s prečno steno. Če stene segajo čez cel vesikel dobimo vesikel razdeljen v več predelov.  Naslednja slika predstavlja vesikel z vzdolžno steno in na koncu imamo vesikel s tubuli, ki imajo enak polmer kor robovi $r$. Razvejišč med tubuli nismo upoštevali, azto je pomembna le skupna dolžina tubulov.

\section{slide 11}
Za občutek kako se vesikli med sabo primerjajo, si lahko pogledamo graf upogibne energije izračunan pri $R..$ in $\eta...$. Vidimo, da vesikel z vzdolžno steno močno izstopa, kar je posledica velike dolžine robov. Na desni slike vidimo, kaj se zgodi če dodamo ADE člen. Vsaki obliki pripadajo parabole in glede na vrednost parametra $\Delta a_0$ so stabilne različne oblike; rdeče območje označuje stabilnost vesikla brez invaginacije, rumeno z eno popolno prečno steno, zeleno z dvema poplnima prečnima stenama,....

\section{slide 12}
predno si ogledamo fazne diagrame, si poglejmo kako smo minimizairali energijo. Tu vidimo celoten izraz za energijo, ki jo sestavljajo trije členi; lokalna up. ener. nelokalna up. ener in medmembranska interakcija. Energijo minimiziramo pri konstantni površini, reduirani prostornini, konstantnem polmeru ograditve $R$ in polnilnem razmerju $\eta$ blizu $1$.

\section{slide 15}
Kaj je pomembno: za mitohondrij sta značilni dve stanji; ortodoksno in kondenzirano, ki imata različne geometrije uvihkov in različno površino membrane ter posledično različne reducirane volumne. Naš model takšno obnašanje pokaže pri velikih vednosti odbojne medmembranske interakcije in večjih vrednosti parametra $q$. Model bi bilo sicer možno v več smereh razširiti in izboljšati; kot npr. uporaba gibke ograditve, večje število perčnih oz. vzdolžnih sten, uporaba bolj splošne oblike invaginacije, ki bi združevala tubularne in lamelarne dele.
	

\end{document}
	