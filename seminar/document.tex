\documentclass[openany, longbibliography,slovene,a4paper,12pt]{article}
%\documentclass[openany,slovene,a4paper,12pt]{article}
\usepackage[a4paper,inner=3.5cm,outer=2.5cm,top=2.5cm,bottom=2.5cm]{geometry}

\usepackage{braket}
\usepackage{float}
\usepackage{afterpage}
\usepackage{graphicx}
\usepackage{amssymb}

\usepackage[tbtags]{amsmath}
\usepackage[T1]{fontenc}
\graphicspath{{../slike/}{../slike_vezikel_z_robom/}{/home/jure/sola/magisterij/uporabljene_slike/}
{../eps_pdf/}}
\DeclareGraphicsExtensions{.eps,.jpeg,.png,.gif,.pdf}
\usepackage[outdir=./slike/]{epstopdf}
\epstopdfsetup{
	suffix=,
}
\usepackage[multidot]{grffile}

% \usepackage[slovene]{babel}      % slovenski delilni vzorci (!)
% \usepackage[english]{babel}
\usepackage[utf8]{inputenc}
\usepackage{makeidx}
\usepackage{enumerate}
\usepackage{caption}
\usepackage{subcaption}
\usepackage[tbtags]{mathtools}

\usepackage[section]{placeins}

\usepackage[hyphens,spaces,obeyspaces]{url}
\usepackage{breakurl}


\usepackage{ragged2e}
\edef\UrlBreaks{\do\-\UrlBreaks}

\usepackage{makeidx}
\pagestyle{headings}
\makeindex
\usepackage{fancyhdr}
\usepackage[titletoc,title]{appendix}


\usepackage[sort, numbers]{natbib}
\usepackage[pdfa]{hyperref}
\usepackage[x-1a]{pdfx}
\usepackage{pdfpages}
\usepackage{breqn}


\DeclareMathOperator{\arcsinh}{arcsinh}

\def\epsfg#1#2{\epsfig{file=#1.eps,width=#2}}
\def\legendamp#1#2{\vbox{\hsize=#1\caption{\small #2}}}

\setcounter{topnumber}{4}
\setcounter{bottomnumber}{4}
\setcounter{totalnumber}{5}
\renewcommand{\topfraction}{0.99}
\renewcommand{\bottomfraction}{0.99}
\renewcommand{\textfraction}{0.0}
\setlength{\tabcolsep}{10pt}
\renewcommand{\arraystretch}{1.5}

\def\bi#1{\hbox{\boldmath{$#1$}}}
\let\oldvec\vec
\def\vec#1{\mbox{\boldmath$#1$}}
\def\pol{{\textstyle{1\over2}}}
\def\svec#1{\mbox{{\scriptsize \boldmath$#1$}}}

\newcommand{\dif}{\mathrm{d}}
\usepackage{xparse}
\DeclareDocumentCommand{\myint}{o m o o}  
{%
	\int \IfValueT{#1}{#1} \dif #2 \IfValueT{#3}{\dif#3} \IfValueT{#4}{\dif#4}
}
\newcommand{\Alpha}{A}
\newcommand{\Beta}{B}
\newcommand{\Epsilon}{E}
\newcommand{\Kappa}{K}


\begin{document}



\section{Introduction}

Organo-metallic frameworks or MOFs are materials consisting of one or multiple
central metallic ions and surrounding organic ligands.  They are crystalline
materials with very high porosity. Internal surface area in some cases exceeds
6000 $\mathrm{m}^2/{g}$ \cite{introd_to_metal_organ_frameworks}. Thanks to their
chemical properties they are widely used in the fields of biochemistry,
catalysis and electrochemistry. They are especially useful in clean energy
application as storage for gasses (eg. hydrogen) or energy through
absorption/adsorption process \cite{introd_to_metal_organ_frameworks}. They can also be used as gas separation medium, as second harmonic generators in nonlinear optics, some of them also display interesting ferroelectric properties. Their usage is so broad thanks to numerous combinations of metallic ions and organic ligands
 \cite{introd_to_metal_organ_frameworks}  \cite{Assignment_of_Solid_State}. Many
 chemical properties stem from unpaired electrons, which are often found in such
 materials. Ions commonly found in MOFs are Cu(||), Ni(||), Pb(|||). Such ions
 have unpaired electron(s) in their d orbitals and one can clearly see their
 effects on $^{13}C$ and $^{1}H$ nmr  spectra, one of the most common tools for
 characterization of the molecular and electronic structure of organic
 molecules. Materials presented in this work are paramagnetic. This means they exhibit weak attraction to the external magnetic field, which is a consequence of unpaired electrons in their structure. The effects of the latter are easily recognized in nmr spectra by the large paramagnetic shifts they cause.

 \begin{figure}
   \centering
      \begin{subfigure}[b]{0.5\textwidth}
  \centering
  \includegraphics[width=0.65\textwidth]{hkust_molekula_placeholder.png}.
  \caption{}
\end{subfigure}%
   \begin{subfigure}[b]{0.5\textwidth}
  \centering
  \includegraphics[width=0.9\textwidth]{hkust_spekter.png}.
  \caption{}
\end{subfigure}
  \caption{ An example of mof material named Hkust-1. It is a crystalline
    material. Image (a) depicts a part of the structure used for calculation of
    nmr parameters of atoms marked with 1, 2 and 3. (b) $\mathrm{C}^{13}$-nmr
    spectra of atoms marked on image (a).
  }
  \end{figure}

 Nmr spectra of organic materials usually feature chemical shifts caused by
 induced currents which in turn are caused by external magnetic field
 \cite{chemic_shift_tensor_review}. In paramagnetic materials, this is not the
 only contribution to the total shifts visible in spectra. An important
 interaction, not present in  diamagnetic materials, is interaction between
 unpaired electrons and nuclei. Such interaction can cause large paramagnetic
 shifts also called hyperfine shifts. Typical for such spectra are also widened
 spectral lines. These large shifts make interpretation of spectra more difficult
 \cite{Dft_Investigation_of_the_Effect_of_Spin_Orbit}. A useful tool to help
 with the interpretation are first-principle quantum calculations. Large growth
 of computational power in recent years have enabled more accurate calculations
 and calculations performed on a more complex systems. However, calculations of
 hyperfine constants and total chemical shifts for paramagnetic materials / mofs
 have not yet been systematically tested and documented in literature.

 In this work we will present the most common approach to calculation of nmr
 parameters of paramagnetic materials,  which is based on density functional
 theory (dft). First we  will present basics of dft, which is the most important part of the whole procedure. Values of nmr parameters completely depend on electronic configuration, consequently it's accurate calculation is of a paramount importance. 

\section{Calculation of electronic wavefunction}
Accurate calculation of electronic structure has always been a challenge. It
quickly became apparent that direct use of Schr{\"o}dinger equation is not a
realistic prospect, except for some small
molecules, as the time consumed to solve it grows exponentially
\cite{nobel_lecture} as a function of electron
number at a given accuracy level. With the development of computers, different
numerical approximations for computation of electronic wave function and
optimization of molecular structure have emerged. One of the most successful methods has been density functional theory
(dft from now on), which has been known for roughly 50 years. Through the years dft has evolved and today it represents one of the main tools for calculation of electronic structure especially for complex molecules and crystals.

\subsection{Hamilotnian}
The first step in formulation of the problem is to define hamiltonian which
describes motion of nuclei and electrons. Non-relativistic hamiltonian
describing the interaction of  nuclei and electrons can be written as follows:
\begin{equation} \label{full_hamiltonian}
\hat H= \hat T_n + \hat  T_e + \hat  W_{n-n} + \hat W_{e-e} + \hat W_{n-e} + \hat V_{ext},
\end{equation}
where $T_n$ and $T_e$ are kinetic energies of nuclei and electrons respectively,
$W_{n-n}$, $W_{e-e}$ and $W_{e-n}$ represent  nuclei-nuclei, electron-electron
and electron-nuclei interaction terms. $V_{ext}$ is strictly multiplicative
external potential. Ground state of such a system is given by the solution of
the time independent Schr{\"o}dinger equation:
\begin{equation} \label{ham_solution}
\hat H \psi = \epsilon_0 \psi_0,
\end{equation} 
where index $0$ denotes the solution with the lowest energy. In general $\psi_0$
depends on $3N$ coordinates, where $N$ is total number of particles. This means
that systems with more than e.g. 10 atoms are very computationally
demanding. It is common practice to reduce the dimensionality of the problem by employing
Born-Oppenheimer approximation in which the motion of nuclei is separated from
motion of electrons, sometimes nuclei are even fixed. Thus, from now on we will
restrict ourselves to hamiltonians describing only the motion of electrons:
\begin{equation} \label{electron_hamiltonian}
\hat H=  \hat  T_e  + \hat W_{e-e} + \hat W_{n-e} + \hat V_{ext}.
\end{equation}
The number of electrons $N$ for a small molecules, like water, is of the order
$\sim 10$. In medium sized molecules with $\sim 50-100$ atoms, the number can
grow to a few hundred, while in large molecules, like proteins, the number can
grow into thousands and ten-thousands. As one can imagine, solving a system of
coupled differential equations with such huge number of coordinates ($3N$) is
computationally an almost impossible task.
This is the reason for development of approaches which, while still being
sufficiently accurate, offer faster computational times. DFT represents one of
the most successful methods to solving such systems.

\subsection{Density functional theory - dft}
Dft is a method, which allows us to map many-electron problem to an effectively
single electron problem. It replaces all electron wave function with
particle density.  The core of dft lies in Kohn-Sham theorems. These two theorems
ensure that stationary many-particle systems are fully characterized by their
ground state particle density. This means that given a group of electrons, one
only has to know ground state electron density to be able to tell all other
properties of the system. For non-degenerate case the latter is uniquely
determined by ground state many-particle wave function, which in turn is
uniquely determined by the external potential. For a simple non-degenerate case
we will prove these two theorems.
Let us consider hamiltonian of the form:
\begin{equation} \label{ks_hamiltonian}
\hat H = \hat T + \hat W + \hat V_{ext},
\end{equation}
where $T$ is kinetic energy, $W$ is inter-particle interaction and $V_{ext}$ is
external potential determined up to a constant. Let $V_{ext}$ be such potential
that ground state $\psi_0$ is non-degenerate. Consider now the set of all $H$ of
the form (\ref{ks_hamiltonian}), which differ only in $V_{ext}$, with
non-degenerate ground states. Since kinetic and inter-particle interaction terms
are the same for all $H$ in such set, the latter can be represented by the set of all non-equivalent potentials:
\begin{equation}
  \begin{split}
    \mathcal{V} = \{V_{ext};\quad &\textrm{V determined up to multiplication factor and a constant;}\\
    & \textrm{$\ket{\psi_o}$ exists and is non-degenerate}\}
    \end{split}
 \end{equation}
According to the above definition we can define the set of all corresponding
ground state-densities determined up to phase as:
\begin{equation}
  \begin{split}
    \mathcal{G} = \{\psi_0; \quad &\textrm{$\psi_0$ ground state corresponding to a potential from $V$;}\\
&\textrm{ $\psi_0\sim\psi_0e^{i\phi}$   }
    \}
    \end{split}
  \end{equation}
The map from $\mathcal{V}$ to $\mathcal{G}$ is surjective by definition. What we
would like to know is, if it is also injective  (\ref{bijection}), i.e., can a
single $\psi_0$ be a ground state for two non-equivalent potentials? Suppose now
that $\psi_0$ is a ground state for two non-equivalent potentials $V_{ext}$ and $V'_{ext}$.

\begin{figure}[!ht]
  \centering
  \includegraphics[width=0.6\textwidth]{bijekcija_med_v_psi_n.png}.
  \caption{Bijection between the set of potentials, their corresponding ground
    states and ground state densities. Existance of such bijection is proven by
    Kohn-Sham theorems and proves that many-particle system is
    uniquely determined by it's ground state particle
    density~\cite{advanced_course}.}
  \label{bijection}
\end{figure}

\begin{dgroup*}
\begin{dmath}
 \hat H\ket{\psi_0} =(\hat T + \hat W + \hat V_{ext}) \ket{\psi_0}\hiderel{=}\epsilon_0\ket{\psi_0}
\end{dmath},
\begin{dmath}
 \hat H'\ket{\psi_0} =(\hat T + \hat W + \hat V'_{ext}) \ket{\psi_0}\hiderel{=}\epsilon'_0\ket{\psi_0}
\end{dmath},
\begin{dsuspend}
subtracting above equation yields:
\end{dsuspend}
\begin{dmath}
(\hat V_{ext} - \hat V'_{ext})\ket{\psi_0}=(\epsilon_0-\epsilon'_0)\ket{\psi_0}
\end{dmath}.
\begin{dsuspend}
 due to multiplicative nature of potentials, we can just divide the whole
 equation by $\ket{\psi_0}$ and obtain: 
\end{dsuspend}
\begin{dmath}
  (\hat V_{ext} - \hat V'_{ext})=(\epsilon_0-\epsilon'_0),
  \end{dmath}
\end{dgroup*}
which is a contradiction, since $V_{ext}$ and $V'_{ext}$ must differ for more
than a constant. Similarly one can show that two different ground state wave
functions, corresponding to two different external potentials, cannot lead to
the same ground state densities.  To see this we compare ground state energies
and rewrite them using ground state densities, which are supposed to be the same
for both wave functions:

\begin{dgroup*}
\begin{dmath}
 \bra{\psi_o}\hat H\ket{\psi_0}=\epsilon_0 < \bra{\psi'_o}\hat H\ket{\psi'_0}=
 \bra{\psi'_o}\hat H + \hat V'_{ext} - \hat V'_{ext} \ket{\psi'_0}=\\ \epsilon'_0
 +  \bra{\psi'_o} \hat V_{ext} - \hat V'_{ext} \ket{\psi'_0}.
\end{dmath}
\begin{dsuspend}
 Rewriting this in terms of densities and taking into account equivalence of H
 and H' yields:
\end{dsuspend}
\begin{dmath}
\epsilon_0 <  \epsilon'_0 + \int (V_{ext}(\vec r)-V'_{ext}(\vec r))n(\vec r)
\dif \vec r
\end{dmath}
\begin{dsuspend}
  and
\end{dsuspend}
\begin{dmath}
\epsilon'_0 <  \epsilon_0 + \int (V'_{ext}(\vec r)-V_{ext}(\vec r))n(\vec r)
\dif \vec r.
\end{dmath}
\begin{dsuspend}
  Upon subtracting both equations one obtains a contradiction:
\end{dsuspend}
\begin{dmath}
  \epsilon_0 + \epsilon'_0< \epsilon_0 + \epsilon'_0,
\end{dmath}
\end{dgroup*}
which proves that for hamiltonians which yield non-degenerate ground states,
each ground state leads to a different ground state particle density. Ground
state particle densities form a set where each density corresponds to a single
wave function from $\mathcal G$:
\begin{equation}
  \mathcal N = \left\{n; n=\bra{\psi}\hat n \ket{\psi}, \psi \in \mathcal G \right\},
\end{equation}
where $\hat n$ is quantum mechanical particle density operator.

Extension of this simple proof to hamiltonians with degenerate ground states is
possible by replacing ground state wave functions by linear span of degenerate
ground states. Thus, in degenerate case one obtains bijection between external
potential, set of linear spans, each belonging to a certain external potential
and a set of sets of ground state densities. Special treatment is necessary also
for systems containing magnetic fields, where one can separate hamiltonian into
spin up and spin down hamiltonian of the form (\ref{ks_hamiltonian}). In the
latter case one obtains bijections between the set of pairs $(\vec A(\vec r),
V_{ext}(\vec r))$, $(\psi_\uparrow, \psi_{\downarrow})$ and $(n(\vec r), \vec m
(\vec r))$.

Since there exists bijection between ground state wave functions and ground
state densities, one can formally rewrite ground state wave function as
functional of ground state particle density:
\begin{equation}
  \ket{\psi'_0} =  \ket{\psi'_0[n]}
  \end{equation}
and using above formula one can also rewrite operators in terms of ground state
density. As an example, let us rewrite ground state energy as functional of
ground state particle density:
\begin{equation} \label{hamiltonian_density}
  E[n_0] = \bra{\psi_0[n_0]}\hat H \ket{\psi_0[n_0]},
  \end{equation}

for which one can find minimum energy principle: $E[n_0]<E[n]$ whenever n
belongs to $\mathcal N$. This an obvious consequence of wave function functional
$\ket{\psi [n]}$, which is only defined for densities which are in $\mathcal N$.
Thus, one has to ask himself whether every nonegative normalizable function
$n(\vec r)$ is in $\mathcal N$.  The answer is no. A density from $\mathcal N$
have a corresponding potential $V_{ext}$ such that it minimizes energy functional
of the form \ref{hamiltonian_density} and consequently they are called V-representable densities. 

In practice, one does not need deep knowledge about such mathematical
definitions of functionals. The reason for this is the discretization of space
into grid points. On final grid for any strictly positive particle density
($n(\vec r) > 0$), which is compatible with Pauli principle, there exists a
potential for which the density represents ground state density and is thus contained in $\mathcal N$ \cite{advanced_course}.


  \section{DFT in practice}
Kohn-Sham theorems unfortunately tell nothing about the explicit dependence of
energy functional on density. Nonetheless, we would still like to use Kohn-Sham
theorems, to construct numerical scheme where electron density has the central
role. We will concentrate on  construction of energy functional $E[n(\vec r)]$,
which should approximate $F[n(\vec r)]$ as well as possible.
  
  \section{Kohn-Sham equations}
  Kohn-Sham (KS) equations represent a standard and most common approach to dft.
  They are based on energy functional dependent only on electron density $n(\vec
  r)$. To introduce them in an understandable and consistent fashion, let's start with a non-interacting system. 
 
\subsection{Noninteracting system}
Hamiltonian of $N$-electron non-interacting system can be written in the
following way:
 \begin{equation} \label{noninteracting_H}
   \hat H =\hat  W_k + \hat V_{ext}, 
 \end{equation}
 where $V_{ext}$ is an external potential of multiplicative nature. It is well
 known the solution to this problem can be written in the form of Slater determinant:

 \[
       \ket{\Phi_0} = \frac{1}{\sqrt{N!}}\det 
   \begin{bmatrix}
   \phi_{1}(\vec r_1, \sigma_1) & \phi_{2}(\vec r_1, \sigma_1) & \phi_{3}(\vec
   r_1, \sigma_1) & \dots & \phi_{N}(\vec r_1, \sigma_1) \\
    \phi_{1}(\vec r_2, \sigma_2) & \phi_{2}(\vec r_2, \sigma_2) & \phi_{3}(\vec
    r_2, \sigma_2) & \dots & \phi_{N}(\vec r_2, \sigma_2) \\
    \vdots & \vdots & \vdots & \ddots & \vdots \\
     \phi_{1}(\vec r_N, \sigma_N) & \phi_{2}(\vec r_N, \sigma_N) & \phi_{3}(\vec r_N, \sigma_N) & \dots & \phi_{N}(\vec r_N, \sigma_N) \\
\end{bmatrix},
\]
 which, when inserted into equation (\ref{noninteracting_H}) effectively produces
 single electron problem:
 \begin{equation}
   \left(-\frac{\hbar^2}{2m}\nabla^2 + \hat V_{ext}(\vec r)\right) \psi (\vec r) = \epsilon_i \psi(\vec r).
 \end{equation}
 Thus, using slater determinant we have effectively converted many electron
 problem with $3N$ coordinates to a single electron problem. Of course the
 electrons in this case do not feel each other and motion of each particle
 should not depend on other particles, so such a breakdown is completely natural
 (except for Pauli principle which is already built into Slater determinant).
 
 The ground state of such a system is obtained using $N/2$ lowest states by
 putting $2$  electrons into each state. Calculation of kinetic energy and
 electron density for such a system is also straight forward. As we can see,
 using Slater  determinant as ansatz for the solution of noninteracting
 hamiltonian offers simple expressions for ground state wavefunction, kinetic
 energy and particle density  calculation. Electron density corresponding to
 such a wave function can be written using the following expression:
 \begin{equation} \label{KS_density}
   n_0(\vec r) = \sum_{\sigma=\uparrow,\downarrow}\sum_i\Theta(\epsilon_F-\epsilon_i)|\phi_i(\vec r, \sigma)|^2.
 \end{equation}
 The density of electrons is just a sum over all occupied orbitals. Now let's
 remember Kohn-Sham theorems, which state that ground state density is unique
 functional of the ground state wave function:
 \begin{equation}
   \ket{\psi}= \ket{\psi(n(\vec r))}.
 \end{equation}
 One can show that there exists an even stronger connection; every $\phi_i$ is a
 uniquely determined by ground state density. One can see this by considering a single particle problem using the same potential $V_{ext}$ as found in eq.
 (\ref{noninteracting_H}) and then gradually adding particles, thus:
 \begin{equation}
   \ket{\phi_i}= \ket{\phi_i(n(\vec r))}.
 \end{equation}
 Using this relations we can define HK functional:
 \begin{equation}
   E_s[n(\vec r)] = \bra{\psi[n(\vec r)]}T \ket{\psi(n(\vec r))} + \int \dif \vec r n(\vec r) V_{ext}(\vec r),
   \end{equation}
 where
\begin{equation} \label{KS_kinetic_term}
   T[n(\vec r)] = \sum_i\Theta(\epsilon_F-\epsilon_i)  \sum_{\sigma=\uparrow,\downarrow}\int \phi^*_\sigma(\vec r)\frac{-i\hbar\nabla^2}{2m}\phi_\sigma(\vec r),
 \end{equation}
 where we have implicitly used $\ket{\phi_i(\vec r)} = \ket{ \phi[n(\vec r)] }$.
 In practice this in not necessary, since functions $\phi(\vec r)$ are known -
 they are direct result of calculation.
 
Now we would like to use a similar construction to solve hamiltonian
\ref{hamiltonian_density}. Using solution ansatz in the form of Slater
determinant offers straight forward calculation of kinetic energy and particle
density. Using Slater determinant leads to differential equations for $\phi_k$
wavefunctions. Modern computational packages instead of solving differential
equations use large sets of basis functions, which in the end lead to
over determined set of algebraic equations. Solution is then given by populating
lowest energy orbitals until all electrons are allocated.
 
First we have to ask ourselves if it is at all possible to convert interacting
$N$ particle problem ($3N$ coordinates) to an effectively single particle (3 coordinates)
problem. By this we mean if there exists a noninteracting system that has the
same groud state electron density as original interacting system. It turns out
that for every $N$-electron interacting system with ground state density
$n_0(\vec r)$ there also exists a noninteracting system with exactly the same
ground state density. The question still remains what kind of external potential
one should use to produce the same electron density.

Intuitively, one would expect that potential belonging to effective single
particle hamiltonian should reflect the properties, geometry and potentials found
in given many particle system. Hamiltonian naturally contains kinetic energy
term, but it should also somehow contain inter-particle interactions. Usually
Kohn-Sham hamiltonian consists of kinetic energy term, Hartree inter particle
interaction, external potential and exchange-correlation functional \cite{advanced_course}:
\begin{equation} \label{KS_system}
  E[n(\vec r)]=T[n(\vec r)]] + E_H[n(\vec r)]] + E_{ext}[n(\vec r)]] + E_{xc}[n(\vec r)]].
\end{equation}
Hartree term accounts for Coulomb repulsion:
\begin{equation} \label{hartree_term1}
  E_H[n(\vec r)] = \int n(\vec r)w(\vec r, \vec r') n(\vec r')\dif \vec r' \dif \vec r.
\end{equation}
The above expression is not the same as the one we get from Hartree-Fock
equations and includes self-repulsion. Dft treats exchange term, also arising
from Colulomb potential separately. Computational packages often employ different
approximation and optimizations for faster calculation of this term \cite{orca}.
Functional belonging to external potential is well known:
\begin{equation} \label{ext_potential_functional}
  E_{ext}[n(\vec r)] = \int n(\vec r) V_{ext}(\vec r) \dif \vec r.
\end{equation}
Unfortunately exchange-correlation functional is much less well known. It is
defined by the equation (\ref{KS_system}) and contains all inter-particle
interactions not contained in $T[n(\vec r)]$ and $ E_H[n(\vec r)]]$. Although
the kinetic energy functional has the same form as exact functional, out basis
functions lack correlation between particles, i.e. coordinate couplings
$x_1x_2$,.. The second contribution comes from self-interaction contained in
Hartree term. Thus, one can formally write the difference between
exchange-correlation energy of real system and KS system (without $E_{xc}$ functional) as: 
\begin{equation}
  E_{xc}^{\psi_0}-E_{xc} = \bra{\psi_0}\hat T + \hat W \ket{\psi_0} - \bra{\Phi_0}\hat T \ket{\Phi_0} -  E_H[n(\vec r)],
\end{equation}
where $\ket{\psi_0}$ is the true ground state of interacting system. It is
common to divide $E_{xc}[n(\vec r)]$ into exchange $E_x[n(\vec r)]$ and
correlation $E_c[n(\vec r)]$ part. The former is defined in such a way that
Hartree-Fock ground state density and energy are reproduced, if correlation part
is neglected. This is achieved if exchange part of functional has exactly the
opposite contribution as excessive hartree term in KS equations: 

\begin{equation}
  E_{x}[n(\vec r)]=-\sum_{k,l}\Theta(\epsilon_F-\epsilon_k)\Theta(\epsilon_F-\epsilon_l)\frac{1}{2}\int \dif \vec{r} \dif \vec{r'}\phi^*_k(\vec r, \sigma)\phi_l(\vec{r}, \sigma)w(\vec{r},\vec{r'})\phi^*_l(\vec{r'}, \sigma')\phi_k(\vec{r'}, \sigma')
\end{equation}
The most important contribution of $E_{x}[n(\vec r)]$ is cancellation of
self-repulsion. This term is commonly called exact exchange and is not always
used in dft calculations. Often times exchange functionals soley based on
electron density are employed and they do not manage to completely cancel
self-repulsion terms. We will take a closer look at them in the following chapter.
Correlation term is more difficult to derive and we will not dwell deeper into
it. Instead let's have a look at how KS equations are actually solved.

Our goal is to minimize hamiltonian of the form (\ref{KS_system}) using Slater
determinant as ansatz for many electron system and density calculation. By
considering (\ref{ext_potential_functional}) and (\ref{KS_kinetic_term}) one can
write KS equation:  
\begin{equation}
  \left( \hat T + v_{ext} + E_H[n(\vec r)]]  + E_{xc}[n(\vec r)]] \right) \ket \phi_i =  \epsilon_i \phi_i,  
\end {equation}
where density $n(\vec r)$  has to be calculated according to equation
(\ref{KS_density}). They are usually determined by inserting ansatz in the
form of gaussian basis functions from which one can
then construct correct single particle states. Multiplying equation by $\bra
\phi_j$ from the left side yields generalized eigenvalue problem:
\begin{equation}
  \bra {\phi_j} \left( \hat T + v_{ext}(\vec r) + E_H[n(\vec r)]]  + E_{xc}[n(\vec r)]] \right) \ket{\phi_i} =  \epsilon_i \braket{ \phi_j |\phi_i }.
\end {equation}

System obviously has to be solved in a self consistent fashion
\ref{self_consistent_scheme}. One starts with the positioning of nuclei into
desired positions and construction of nuclei potentials. In parallel with the
last step starting electron orbitals $\phi^{(0)}_i$ are constructed. Usually they are written
as a series of Gaussian functions. Electron density $n^0$ can then be quickly
calculated from constructed orbitals. The next step is to
calculate functionals $E_H[n(\vec r)]$ and $E_{xc}[n(\vec r)]$. In the next step
equations are solved. The solution are new orbitals $\phi_i^{(1)}$  and from
them new density $n^{(1)}$ is calculated. The latter is then used to construct new $E_H[n(\vec r)]$ and $E_{xc}[n(\vec r)]$ which are again used to
solve KS equation from which one gets new orbitals $\phi^{(3)}$. The procedure is
repeated until the change in density or energy is not small enough. 


\begin{figure}
  \centering
  \includegraphics[width=0.65\textwidth]{self_consistent_scheme.png}.
  \caption{Self consistent scheme depicting the procedure in which Kohn Sham
    equations are solved. One starts positioning nuclei into
    desired positions. The next step is to construct potential of nuclei and in
    parallel one can also construct starting approximation for electronic wave
    function. The latter is then used for constructing all other functionals of
    density. After all functionals have been established one can solve KS
    equations, construct new density and repeat the described procedure until
    the change in density is not small enough.
  }
  \label{self_consistent_scheme}
\end{figure}

When given a certain system, functionals $T$, $E_H$ and $V_{ext}$ are precisely
determined. $E_{xc}$ is not. It is thus the determining factor for how well the KS
equations describe our system. We will present various functionals used today in
one of the next chapters.

\subsection{Degenerate ground state}
So far we have talked little about the problem of degeneracy of KS states. In
general degeneracy is not a problem, except at Fermi level. When there are more
KS states than there are electrons, several possible ground states and thus also
particle densities can be constructed. In such cases density matrix arising from
several Slater determinants is constructed:
\begin{equation}
  \hat D = \sum_i d_i\ket{\Phi_i} \bra{\Phi_i}.
  \end{equation}
Density belonging to such ground state is just a weighted sum of densities
corresponding to each Slater determinant $n(\vec r)=\sum_id_in_i(\vec r)$, where
$n_i(\vec r)=\sum_j\phi_{ij}(\vec r)$. Index $i$ runs over all possible
Slater determinants one can construct from given degenerate states and index $j$
runs over all states in each Slater determinant. The sum of coefficients $d_k$
is the trace of density matrix and has to be $1$. The choice of coefficients
$d_k$ is not trivial. They have to be constructed in such a way that the new
particle density does not break degeneracy. When a new degenerate state emerges in
scf procedure (fig. \ref{self_consistent_scheme}), it may significantly affect
particle density and resulting potentials and consequently break or destroy
convergence of the procedure. An example is a boron atom where 2p orbitals are
degenerate. Only the choice of $d_{2p^0}=d_{2p^{-1}}=d_{2p^1}=1/3$ leads to spherically symmetric potential and preservation of degeneracy\cite{advanced_course}.

\subsection{Exchange and correlation functionals}
Exchange and correlation functionals are functionals which try to account for
exchange and correlation interactions between particles/electrons. They can be of a
different forms and in general one can roughly divide them into 4 groups
\cite{challenges_den_fun_theor} - Lda, Gga, meta-Gga and Hybrid functionals.
The first three group all have in common that they explicitly depend on density
of particles. This causes a self-interaction error, which causes excessive
delocalization of electrons. As a solution to these problems hybrid
functionals, which contain exact Hartree-Fock exchange, have been proposed.

\subsubsection{Lda}
Local density approximation (lda) are functionals which depend only on
density of particles (electrons) \cite{challenges_den_fun_theor}.
First functional of such form dates back into
the year 1930, when the exchange interaction for uniform gas was discovered \cite{challenges_den_fun_theor}:
\begin{equation} \label{electron_gas_exchange}
E_{xc}^{lda}[n]=-\frac{3}{4}\left( \frac{3}{\pi} \right)^{1/3}\int n(\vec r)^{4/3} \dif \vec r.
\end{equation}
Today, there exist multiple other lda exchange functionals, but for most cases
they are not very useful. Their only advantage are fast computational times.
For every other purpose gga and hybrid functionals are much better suited.

\subsubsection{Gga}
Generalized gradient approximation (gga) functionals
depend not only on density of particles but also on its gradient \cite{challenges_den_fun_theor}. Most commonly,
gga functionals are build upon (\ref{electron_gas_exchange}) \cite{challenges_den_fun_theor}:
\begin{equation}
  E_{xc}^{gga}=\int n(\vec r)^{4/3}F(x);\quad x=|\nabla n(\vec r)|/n(\vec r)^{4/3}.
\end{equation}
One of most commonly used gga functional is PBE functional
\cite{challenges_den_fun_theor} \cite{challenges_den_fun_theor}:
\begin{equation}
  E_{xc}^{pbe}=-\int  n(\vec r)^{4/3}\left[ \frac{3}{4}\left(\frac{3}{\pi}\right)^{1/3} + \frac{\mu s}{1+ \mu s^2/\kappa} \right]; \quad s=x/(2(3\pi/2)^{1/3}).
\end{equation}
Gga functionals offer acceptable accuracy at fast computational times and are
most commonly used for approximate calculation, before starting more accurate
and more time consuming calcualtion using hybrid or meta-gga functionals.

\subsubsection{Meta-gga functionals}
Since gga functionals have their short commings, meta-gga functionals were formed
in belief that adding higher derivatives will improve accuracy
\cite{challenges_den_fun_theor}. Meta-gga functionals are build upon gga
functional form and add terms containing higher order derivatives of particle
density. Functionals are formulated according to the following equation
\cite{challenges_den_fun_theor}:

\begin{equation}
  E_{xc}^{MGGA}= \int \rho^{4/3}F(\rho(\vec r) \nabla \rho(\vec r), \nabla^2 \rho(\vec r), \tau(\vec r)); \quad \tau=\frac{1}{2}\sum_i |\nabla \phi_i(\vec r)|^2
   \end{equation}

Meta-gga functionals have higher computational cost than gga functionals
Unfortunately, as it turned out, they are not significantly better than gga
functionals and are thus not so popular.


\subsubsection{Hybrid functionals}
On the contrary to meta-gga functionals, hybrid functionals are much more
successful \cite{challenges_den_fun_theor}. These functionals are not a
continuation of lda, gga, meta-gga chain. Instead, they incorporate exact
Hartree-Fock exchange term \cite{challenges_den_fun_theor}:
\begin{equation}
  E^{hf}_{xc}=\sum_ {i,j,\sigma}\int\frac{\phi^*_{i\sigma}(\vec r) \phi_{j\sigma}(\vec r) \phi^*_{j\sigma}(\vec r') \phi_{i\sigma}(\vec r')}{|\vec r - \vec r'|}\dif \vec r \dif \vec r'.
\end{equation}
Such functionals do not break Kohn Sham formalism since the wave functions
$\phi_i(\vec r)$ are unique functional of densities shown in previous chapter.


\begin{figure}[!ht]
  \centering
  \includegraphics[width=0.4\textwidth]{jacobs_functional_ladder_ver2.png}.
  \caption{Jacob’s ladder of exchange-correlation functional approximations
    employed in dft calculations. Hartree world represents starting level where only
    weak interparticle interaction is present. Lda approxiamtion covers
    functionals which depend only on particle density. Gga additionally depends
    on gradient of particle density, while mega-gga incorporates higher order
    derivatives of particle density and in some cases even kinetic energy of
    orbitals ($\tau$). Hyper-gga functionals, also called hybrid functionals,
    stage represents functionals which contain exact exchange calculation (i.e.
    the one found in Hartree-Fock method). The last level utilizes all Kohn-Sham
    orbitals to calculate correlation and exchange interaction. This level also
    accounts for VdW interaction, which is caused by charge fluctuations and is
    not accounted for in previous stages \cite{How_theo_simul_can_address}.
  }
  \label{bijection}
\end{figure}

\subsection{Van der Waals dispersion}
Dft does not take into account effects caused by charge fluctuations. These Van
der Waals effects can be taken care of with different methods.
The most straight
forward way is simply to add a special non-local functional (VdW-DF method). A similar
approach is to use highly parameterized functionals, like metahybrid
functionals (called DFs). Another way is to add dispersion-corrected
atom-centered potentials (DCACPs). Recently, the most promising seems to be DFT-D method
\cite{consis_accur_ab_initio_param}, which is just a sum of terms $CR^{-6}$
over all atom pairs. As it is generally known, for large distances Van der Waals
potential should decay as $R^{-6}$. It is obvious that DFT-D does obey this
rule as also does VdW-DF method. The other two unfortunately do not obey it. As
a consequence DF and DCACP usually cause underbinding.


\section{All electron vs pseudopotential}
Electronic states of an atom can be divided into three categories; core states,
semi-core states and valence states. The latter are the most actively included
in formation of bonds. Valence states may be completely deformed once the atom
is put into molecules/crystals. Semi-core state are states which do not
directly contribute to bonding, but may still be polarized or spatially
deformed. Lastly, core states, are highly localized and are assumed to be
unaffected by chemical bonding. This means that there should be very little loss
of accuracy if core states are replaced by a pseudo wavefunction.
Pseudo potentials are constructs which try to
replicate effects of core electrons exerted on semi-core, valence electrons and
thus reproduce correct chemical and physical properties (bonding energy, bond
length, electron localization, magnetization,...).

Of course, pseudo potentials have to be constructed prior to a given calculation using
some other much simpler system. As a consequence, there
exists a question of transferability. Is a potential constructed using some
reference molecuole usable for another molecule. The exact answer is
not possible. Many times the only way is to try, especially when transition
elements are in question.

All electron calculations avoid pseudopotential by taking into account all
electrons. The latter does cost some computational time, but it depending on
molecule, it may be the only way to get reliable result. Of course for large
molecules all electron calculations are significantly more expensive.
Commonly used open source packages are Orca and Nwchem.


\section{Dft in crystals}
Crystals are periodic structures. This fact is also reflected the shape of electronic  wave functions, which are Bloch states \cite{solid_state_physics}:
\begin{equation}
  \psi_{n,\vec k}(\vec r) =e^{i\vec k\vec r}\sum_G c_{n,\vec k}(\vec G)e^{i\vec G \vec r} \quad \mathrm{with} \quad \psi_{n,\vec k}(\vec r+\vec R)= \psi_{n,\vec k}(\vec r)
\end{equation}
where $\vec G$ is a vector of reciprocal lattice and $\vec R$ vector of bravais lattice of a crystal in question.
In ideal case, where core potentials are neglected, wave functions are simple plane waves. When potential is weak and reasonably smooth, it can be treated as perturbation \cite{solid_state_physics}.
Unfortunately neither of the two conditions is true. Core potential diverges as $r \rightarrow 0$
and as a consequence the wave functions have a cusp at the origin. For heavier
atoms core states have large gradients and can not be represented as plane waves.

The simplest solution is to expand wave functions into series of plane waves.
Computationally this is just a discrete fourier transform.
However, because of already given reasons, the number of plane waves required
for such expansion is very large. Thus, even the sum of plane waves is not a
suitable representation of core states. For this reason, core states are
commonly replaced by pseudo potentials in crystal dft calculations. This
approach usually carries \emph{PW calculation} designation.

Although, for many crystals this approach works well it is not good enough
especially for transition elements with partially filled d-shells and second row
elements. Electron density of transition metals and second row elements still
varies widely in spite of use of pseudopotentials \cite{large_scale_quant_mechan_enzym}. For such cases an improved
approach has been developed. A new method, called gaussian augmented plane waves
or \emph{GAPW} approach. GAPW basis sets consist of Gaussian functions and plane
waves. This approach is suitable for all electron calculations, where core
states are expanded in Gaussian functions and valence electrons in plane waves. As a consequence, all electron wave calculations are possible for crystals avoiding pseudo potential inaccuracy issues.

\section{Nmr parameters}
The two most important nmr quantities are shielding tensor and hyperfine tensor.
The former is important in all organic materials, while the latter is present
only in materials with unpaired electrons.Shielding tensor is 
proportionate to electron density at the observed  core site, whereas hyperfine
tensor is proportionate to shielding tensor.
\begin{minipage}[t]{0.5\linewidth}
  \centering
 \begin{figure}
  \includegraphics[width=0.4\textwidth]{origin_dependance_tensor.png}
  \caption{}
\end{figure}  
\end{minipage}%
\begin{minipage}[t]{0.5\linewidth}
  \centering
   \label{origin_dependence}
\end{minipage}

\section{Conclusion}
In this paper we tried to describe the most important aspects of calculation of nmr parameters.

\newpage \phantomsection


\addcontentsline{toc}{section}{Literatura}
\bibliographystyle{../myapsrevSLO}
\bibliography{../mybib}

\end{document}